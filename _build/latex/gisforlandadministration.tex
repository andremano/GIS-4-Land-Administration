%% Generated by Sphinx.
\def\sphinxdocclass{report}
\documentclass[letterpaper,10pt,english]{sphinxmanual}
\ifdefined\pdfpxdimen
   \let\sphinxpxdimen\pdfpxdimen\else\newdimen\sphinxpxdimen
\fi \sphinxpxdimen=.75bp\relax

\PassOptionsToPackage{warn}{textcomp}
\usepackage[utf8]{inputenc}
\ifdefined\DeclareUnicodeCharacter
% support both utf8 and utf8x syntaxes
  \ifdefined\DeclareUnicodeCharacterAsOptional
    \def\sphinxDUC#1{\DeclareUnicodeCharacter{"#1}}
  \else
    \let\sphinxDUC\DeclareUnicodeCharacter
  \fi
  \sphinxDUC{00A0}{\nobreakspace}
  \sphinxDUC{2500}{\sphinxunichar{2500}}
  \sphinxDUC{2502}{\sphinxunichar{2502}}
  \sphinxDUC{2514}{\sphinxunichar{2514}}
  \sphinxDUC{251C}{\sphinxunichar{251C}}
  \sphinxDUC{2572}{\textbackslash}
\fi
\usepackage{cmap}
\usepackage[T1]{fontenc}
\usepackage{amsmath,amssymb,amstext}
\usepackage{babel}



\usepackage{times}
\expandafter\ifx\csname T@LGR\endcsname\relax
\else
% LGR was declared as font encoding
  \substitutefont{LGR}{\rmdefault}{cmr}
  \substitutefont{LGR}{\sfdefault}{cmss}
  \substitutefont{LGR}{\ttdefault}{cmtt}
\fi
\expandafter\ifx\csname T@X2\endcsname\relax
  \expandafter\ifx\csname T@T2A\endcsname\relax
  \else
  % T2A was declared as font encoding
    \substitutefont{T2A}{\rmdefault}{cmr}
    \substitutefont{T2A}{\sfdefault}{cmss}
    \substitutefont{T2A}{\ttdefault}{cmtt}
  \fi
\else
% X2 was declared as font encoding
  \substitutefont{X2}{\rmdefault}{cmr}
  \substitutefont{X2}{\sfdefault}{cmss}
  \substitutefont{X2}{\ttdefault}{cmtt}
\fi


\usepackage[Sonny]{fncychap}
\ChNameVar{\Large\normalfont\sffamily}
\ChTitleVar{\Large\normalfont\sffamily}
\usepackage[,numfigreset=1,mathnumfig]{sphinx}

\fvset{fontsize=\small}
\usepackage{geometry}


% Include hyperref last.
\usepackage{hyperref}
% Fix anchor placement for figures with captions.
\usepackage{hypcap}% it must be loaded after hyperref.
% Set up styles of URL: it should be placed after hyperref.
\urlstyle{same}

\addto\captionsenglish{\renewcommand{\contentsname}{Using QGIS}}

\usepackage{sphinxmessages}
\setcounter{tocdepth}{2}



\title{GIS for Land Administration}
\date{Jul 18, 2020}
\release{1}
\author{Andre Mano}
\newcommand{\sphinxlogo}{\vbox{}}
\renewcommand{\releasename}{Release}
\makeindex
\begin{document}

\ifdefined\shorthandoff
  \ifnum\catcode`\=\string=\active\shorthandoff{=}\fi
  \ifnum\catcode`\"=\active\shorthandoff{"}\fi
\fi

\pagestyle{empty}
\sphinxmaketitle
\pagestyle{plain}
\sphinxtableofcontents
\pagestyle{normal}
\phantomsection\label{\detokenize{index::doc}}



\chapter{Using QGIS}
\label{\detokenize{using_qgis:using-qgis}}\label{\detokenize{using_qgis::doc}}
This exercise will introduce some of the basic features of QGIS. Along the way you will also become more familiar with QGIS.


\section{About QGIS}
\label{\detokenize{using_qgis:about-qgis}}
QGIS (\hyperref[\detokenize{using_qgis:qgis-logo}]{Fig.\@ \ref{\detokenize{using_qgis:qgis-logo}}}) QGIS is a free and open source software you can download from \sphinxhref{http://www.qgis.org/}{www.qgis.org}. and distribute through as many computers as you wish.

\begin{figure}[htbp]
\centering
\capstart

\noindent\sphinxincludegraphics[scale=0.5]{{qgis_logo}.png}
\caption{QGIS logo}\label{\detokenize{using_qgis:id1}}\label{\detokenize{using_qgis:qgis-logo}}\end{figure}

It is handy to bookmark the official \sphinxhref{https://docs.qgis.org/testing/en/docs/user\_manual//}{QGIS documentation}.


\section{Exercise}
\label{\detokenize{using_qgis:exercise}}
\begin{sphinxadmonition}{note}{Resources}

\begin{DUlineblock}{0em}
\item[] For this exercise you will need this \sphinxhref{https://canvas.utwente.nl/files/1756885/download?download\_frd=1/}{dataset}. The dataset contains the following layers:
\end{DUlineblock}
\begin{itemize}
\item {} 
\sphinxstyleemphasis{admin\_regions.gpkg} (vector layer containing the administrative regions of Guyana)

\item {} 
\sphinxstyleemphasis{rivers.gpkg} (vector layer containing the main rivers of Guyana)

\item {} 
\sphinxstyleemphasis{roads.gpkg} (vector layer containing the main roads of Guyana)

\end{itemize}
\end{sphinxadmonition}
\begin{enumerate}
\sphinxsetlistlabels{\arabic}{enumi}{enumii}{}{.}%
\item {} 
\sphinxstylestrong{Task} Start QGIS and add the three layers

\item {} 
\sphinxstylestrong{Task} Go to \sphinxcode{\sphinxupquote{Plugins \textgreater{} Manage and install plugins...}} the plugin \sphinxstyleemphasis{Quick map services} and add one base map to your project.

\item {} 
\sphinxstylestrong{Task} In the \sphinxcode{\sphinxupquote{Layer\textquotesingle{}s panel}} create two groups (i.e. folders): one called \sphinxstyleemphasis{artifitial} and another one called \sphinxstyleemphasis{natural}

\item {} 
\sphinxstylestrong{Task} Drag the \sphinxstyleemphasis{admin\_regions} and \sphinxstyleemphasis{roads} into the group \sphinxstyleemphasis{artificial} and the rivers layer into the group \sphinxstyleemphasis{natural}.

\item {} 
\sphinxstylestrong{Task} Re\sphinxhyphen{}order the groups and the layers inside in such a way that all the layers are visible.

\item {} 
\sphinxstylestrong{Task} Save your QGIS project and share it with a colleague.

\item {} 
\sphinxstylestrong{Task} Open a project created by a colleague.

\item {} 
\sphinxstylestrong{Task} Re\sphinxhyphen{}arrange one or two toolbars in a way that it is convenient.

\item {} 
\sphinxstylestrong{Task} Make sure the \sphinxcode{\sphinxupquote{Processing panel}} is visible

\end{enumerate}


\chapter{Vector representations}
\label{\detokenize{vector_representations:vector-representations}}\label{\detokenize{vector_representations::doc}}
A common way to represent real world objects in a GIS is using a vector data model to do it.
This exercise will introduce the basic concepts and rationale behind the vector model.

\begin{sphinxadmonition}{note}{Resources}

\begin{DUlineblock}{0em}
\item[] For this exercise you will only need a sheet of paper and a pen/pencil. You are encouraged to work together with a colleague.
\end{DUlineblock}
\end{sphinxadmonition}


\section{Basics of the vector model}
\label{\detokenize{vector_representations:basics-of-the-vector-model}}
The vector model uses the most basic geometric primitive \sphinxhyphen{} the point, to construct more complex geometries like lines and polygons (\hyperref[\detokenize{vector_representations:the-vector-data-model}]{Fig.\@ \ref{\detokenize{vector_representations:the-vector-data-model}}}).
Each point is defined by its X and Y coordinates in the case of 2D geometries and also a Z coordinate in the case of 3D geometries.

\begin{figure}[htbp]
\centering
\capstart

\noindent\sphinxincludegraphics[scale=0.5]{{vector_data_model}.png}
\caption{The vector data model}\label{\detokenize{vector_representations:id1}}\label{\detokenize{vector_representations:the-vector-data-model}}\end{figure}

The coordinates associated to the vector representations are anchored to a Coordinate Reference System (CRS), but this exercise is not covering CRS.
With these geometric constructs, one can abstract real world phenomena into computer representations (\hyperref[\detokenize{vector_representations:abstracting-objects}]{Fig.\@ \ref{\detokenize{vector_representations:abstracting-objects}}}).

\begin{figure}[htbp]
\centering
\capstart

\noindent\sphinxincludegraphics[scale=0.5]{{abstracting_objects}.png}
\caption{Abstracting objects}\label{\detokenize{vector_representations:id2}}\label{\detokenize{vector_representations:abstracting-objects}}\end{figure}

Along with the geometries, when abstracting real world objects or phenomena into a vector model, often we need to provide additional information
as to what those geometries represent, and how they are described. This set of information is known as the \sphinxcode{\sphinxupquote{Atribute table}}. Consider the example of the previous figure,
Can we know more about that road? The answer to that question lies on the attribute table associated with the geometry. That attribute table could be something like this:


\begin{savenotes}\sphinxattablestart
\centering
\begin{tabulary}{\linewidth}[t]{|T|T|T|}
\hline
\sphinxstyletheadfamily 
Name
&\sphinxstyletheadfamily 
one\sphinxhyphen{}way
&\sphinxstyletheadfamily 
length(m)
\\
\hline
M43
&
no
&
7895
\\
\hline
\end{tabulary}
\par
\sphinxattableend\end{savenotes}


\subsection{Multipart geometries}
\label{\detokenize{vector_representations:multipart-geometries}}
A special, but very common case, occurs when you have an object that is composed of two or more parts. These parts form one single object and as such
in the attribute table only one entry will show. One classic example are multi island countries like Indonesia. Each island can be represented as a polygon, but all those polygons compose one single object (\hyperref[\detokenize{vector_representations:indonesia}]{Fig.\@ \ref{\detokenize{vector_representations:indonesia}}}).

\begin{figure}[htbp]
\centering
\capstart

\noindent\sphinxincludegraphics[scale=0.5]{{indonesia}.png}
\caption{Indonesia}\label{\detokenize{vector_representations:id3}}\label{\detokenize{vector_representations:indonesia}}\end{figure}


\section{Exercise}
\label{\detokenize{vector_representations:exercise}}
Now that you are familiar with the vector data model, we invite you to abstract any real world object, man made or natural, into a vector data model.
\begin{enumerate}
\sphinxsetlistlabels{\arabic}{enumi}{enumii}{}{.}%
\item {} 
\sphinxstylestrong{Task} Fill in the following table with any example you consider adequate:


\begin{savenotes}\sphinxattablestart
\centering
\begin{tabulary}{\linewidth}[t]{|T|T|T|}
\hline
\sphinxstyletheadfamily 
Real world object
&\sphinxstyletheadfamily 
represented by a
&\sphinxstyletheadfamily 
attributes(name at least 3)
\\
\hline
\sphinxstyleemphasis{restaurant}
&
\sphinxstyleemphasis{Point}
&
\sphinxstyleemphasis{name; rating; capacity}
\\
\hline
...
&
...
&
...
\\
\hline
...
&
...
&
...
\\
\hline
...
&
...
&
...
\\
\hline
\end{tabulary}
\par
\sphinxattableend\end{savenotes}

\begin{DUlineblock}{0em}
\item[] 
\end{DUlineblock}

\item {} 
\sphinxstylestrong{Task} Provide at least two examples of objects that in principle most be represented by multi\sphinxhyphen{}part geometries

\end{enumerate}


\chapter{Indices and tables}
\label{\detokenize{index:indices-and-tables}}\begin{itemize}
\item {} 
\DUrole{xref,std,std-ref}{genindex}

\item {} 
\DUrole{xref,std,std-ref}{modindex}

\item {} 
\DUrole{xref,std,std-ref}{search}

\end{itemize}



\renewcommand{\indexname}{Index}
\printindex
\end{document}